\begin{abstract}
We present a control-theoretic viewpoint on the three-dimensional incompressible Navier--Stokes equations through the \emph{Adaptive Resistance Principle} (ARP), an adaptive law originally developed to stabilize complex dynamical systems.
The Clay Millennium problem asks whether smooth, globally-defined solutions exist for all smooth divergence-free initial data and body forces.
Our contribution is twofold: (i) an \emph{analytical lens} where an auxiliary ARP state $G(t)\ge 0$ driven by an intensity functional $\Phi(u)$ augments the classical energy method, yielding a Lyapunov-like quantity $V(t)=E(t)+\lambda G(t)$ with tunable dissipation; and (ii) a \emph{numerical framework} in which ARP-inspired adaptation stabilizes GPU-accelerated vorticity-based solvers at high Reynolds numbers.
This manuscript does not claim a resolution to the Millennium problem; instead, it proposes a tractable route for deriving a priori bounds and reports preliminary computational behavior consistent with enhanced stability.
\end{abstract}